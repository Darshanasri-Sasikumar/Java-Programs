Java Features
1.Java is platform Independent Programming Languvange
2.Simple and Secured in Nature 
3.It is an Object Oriented Programming Languvange
4.Java Supports Muliti-Threading
5.Robust in Nature((strong typing, exception handling, automatic memory))

why Java is Platform Independent Languvange
1.Java Supports WORA(Write Once and Run AnyWhere)
2.Any Program that is Return in Java can be executed in any Platform
3.Java is Independent Languvange but Jvm is dependent One
4.Here Executable file is .class which contains byte code

Compiler 
1.Compiler work is to check Java code and then Conver it into byte Code
2.It is used to List Syntax Mistakes
3.The compiler (javac) converts it into a .class file that contains bytecode.(bytecode is nothing but a special intermediate code)
4.It checks the entire program before running it.
5.It can make a program Compile Time Sucess (CTS) and Compile time Error (CTE)

Interpreter
1.JVM is Interpreter
2.It Converts Bite Code into Binary Code

JDK(Java Development Kit)
1.It contains two Important Documents
2.It contains TWO Important Components ie.. Development Tools and JRE(Java Run Time Environment)
3.JVM is Present inside JRE
4.JVM Job is to Convert Byte Code into binary

Commands
Compilation command-->javac FileName.java
Execution command-->java ClassName

Java Memory Areas (Very basic introduction)

Stack
Heap
Method Area
Class Area

Java Architecture Flow
Java Source Code → .java file
Compiler (javac) → converts to .class (bytecode)
JVM → Converts Bytecode to Machine Code (Binary)
OS → Executes Machine Code


Tokens:
In Java, tokens are the smallest meaningful units of a program.
There are 5 types of Tokens:
          1.Keywords
          2.identifier
          3.Operators
          4.literals
          5.special symbols


Keywords
1.keywords are reserved words
2.It is Predefined Words and also called as reserved Word
3.We have 49+ keywords
4.some of them are public,static,int,double,void,final,abstract etc.....

identifier
1.In java we are having components such as class,variable,methods ,interface called as identifiers
2.it is used for identifyin class,method,variable or interface
Rules of Identifeirs:
    1.Can't Start with Numbers But we can Have Numbers inbetween or Last
    2.Except $ and _ can't use other Special Characters
    3.Wide spaces or not allowd
    4.Keywords can't be named for Identifeirs

Operators
1.it is Pre-defined symbols to perform Operations
2.List of Operators
  1.Arithmatic Operators
  2.TypeCast Operators
  3.Relational Operator
  4.Logical Operators
  5.Conditional Operators
  6.Increament and Decreament Operator
  7.Compound Assignment Operator

literals
1.Literals are nothing but Fixed values that can be written in java programming Languvange
2.The literal is the actual value stored inside the variable

Data Types (Primitive & Non-Primitive)
1.Primitive Data Types
2.byte
3.short
4.int
5.long
6.float
7.double
8.char
9.boolean

Non-Primitive Types
String
Array
Class
Interface
Enum

Type Casting
1.Process of Converting one datatype into Another datatype is TypeCastine
We have two types of type casting
Primitive Type casting and Non -Primitive Type casting

Primitive Casting:
   Converting one primitive data type into another primitive data type is know as Primitive data type is know as Primitive type casting
   We have 2 types of Primitive type casting:
         1.Widening
         2.Narrowing
    


    Widening:
           The Process of converting Lower range of primitive datat type into higher range of primitive data type is called widening
           it is an Implicit process
           We can achive widening by typecaste operator or non type caste operator

           Example:
           public class widening {
                  public static void main(String[] args) {
                   // example for widening
                    int num = 10;
                    long number = num;
                    float no = number;
                    System.out.println(no);
                 }
        }

    Q1.Can we achive widening Explicitly?
    ->yes

    Narrowing:
            The Process of converting higher Range of Primitive types into Lower Range of Prmitive types is known as Narrowing
            In Narrowing we can face Data Loss
            so we can expect Compile time error
            We can Achive Narrowing only by type caste operator

            Example:
             public class widening {
                  public static void main(String[] args) {
                   // example for widening

                      double a =10.00;
                      int b = (int)a;

                     System.out.println(b);
                   
                 }
               }

Non-Primitve Type casting:
      1.upcasting 
      2.Down Casting

-----------------------------------------------------------------------------------------------------------------------------------------------

Control Flow statements:
    Controlling the flow of execution is know as Control flow Stmts

we have 2 type of Control flow statements:
        1.Decision making statements->if,else,else if,else if ladder and switch(Examples in Number Programming)
        2.Looping statements->while,do while,for loop advance foreach loop

switch->we cannot pass long ,float ,boolean,double type of data type we can face data loss


Break:
   1.it is a keyword
   2.used to exist from loop
   3.it stops from further iterations

Continue:
   1.it is keyword
   2.It is used to skip the current iteration of a loop and continue with the next iteration.

Return:
  1.End a method and send a value back
  2.it is a keyword


  Method:

   It is A block of instruction it is used to Perform a task
   A method is a block of code which runs when it is been called
   Methods are used to perform certain actions, and they are also known as functions.
   A method can also be called multiple times:
      ->Code Reusability
      ->Readability
      ->better Programming structure
      ->Reduces code length
      ->Easy debugging
   Syntax:
   Access Modifier,Modifier,return type,Method Name,([Data type one and Data type 2])

  We have 2 types of Method:
      1.Parameterized Method
      2.Non-Parameterized Method
  


  Access Modifier:
         This decideds the visibility of the mothod
                    1.public ->all
                    2.private ->within the same class
                    3.default ->within package
                    4.protected->same package but sub class of other package
  
  Modifier:
         These are components which decideds the characterstics of method
         1.static ->belongs to class no Object required
         2.final ->Cannot be ovveriden
         3.Abstract -> No body must be Override
         4.Synchronized->Thread safe
   
   Return Type:
         Specifies what kind of data can be written
         1.void->no return required
         2.Primitive Data type->int ,byte,float,long,double
         3.Non - primitive data type ->string,class,Arrays
   
   return Keyword
      Used to return data to the caller.
      Transfers the control back to the calling method.
      It is Also called as control tranfer Statement
      If written before last line → other lines become unreachable.
   

Can I use return keyword inside constructor?
   yes but it Does Not return Any value



Parameterized Method
----------------------
             Information that can be passed inside a method is called Parameterized Method
             we can pass n parameter by seperating them with comma
             ex:
                public static method(String name){
                  System.out.println("Enter the name" + name);

                }


                public static void main(String[]args){
                  method();
                  
                }



   Types of Methods:
   1.No argument method:
     ex:
        public class method (){}
   
   2.Parameterized method: 
     ex:
        public class void method(int a , int b){}

   3.Static method:
      ex:
        static void show(){}

   4.Abstract method:
      ex:
        void method(){  }
   
   5.final method:
      ex:
       final void method(){}

Final:
final variable → value cannot change
final method → cannot override
final class → cannot be inherited



Explain Method call process:
     When ever method is Called Execution of the Prog Pause
     Control Tranfered to the Called method 
     Execution of Called Method Begins and Execute Completely
     Once Execution completed control is given back to the Caller


what is method overloading?
A class having same name but differnt formal argument is know as method overloading

    ex:
       public static add(int a, int b){
         Syste.out.println(a+b);
       }

       public static add(int a, int b , int c){
         System.out.prinln(a+b);
       }
       


Recursion:
    It is the process of calling itself know as Recursion
    there is a possibility for Run tim error i.e stack overflow Error
    By Providing finite call we can avoid this


    Constructor (Basics)
    -----------------------

               Constructor:
                           It is a Special Type of block Having name similar to className
                           It is used to Load All the non-static mem into Obj
                           Constructors will not have modifiers and return Type
                           Constructor is a special method that is Automatically called when an object is created


               Properties of Constructor:
                                 Same name as class
                                 No return type
                                 Automatically called when object is created
                                 Used to initialize instance variables

              1. Why do we need a constructor?
                           To give initial values to an object
                           Without constructor, variables get default values

               2.Constructor can be static ? Or Constructor can be final?
                  No ,bcz it will not have Modifiers

               3.Can we have private Constructor?
                  Yes,We Can Have But Obj Creation is Only Possible within the class 



   Constructor chainig:
   --------------------
                     one constructor calling another constructor is called constructor chainig
                     we can achive this using
                          ->this call stmt
                          ->super call stmt

               This call stmt:
                        it is used to call the constructor of the same class
                        this call and super call stmt cannot be called together
                        We should use this call stmt inside the constr body as first inst
                        Recursive call is not allowe


               Super call stmt:
                        it is used  to call the constructor from parent class to child class
                        We should use super call stmt as first inst inside the const of child class
                        we want super call stmt to load parent class non-static mem into child class
                        
                        
                        

                        


               Types:
                  Default constructor
                  Parameterized constructor
                  Copy constructor (Java doesn’t provide but can be created manually)

               Default Constructor
               --------------------
                           A constructor with no parameter
                           ->either we should create manually 
                           ->java provides itself(by default java has constructor)

               Parameterized Constructor
               --------------------------
                            A constructor with parameter

                            Why parameterized constructor?
                            ->Avoid writing setter
                            ->more secure and  clean codes
                            -> Force Object to be created with values
               
               Constructor overloading
               ------------------------
                             A constructor having same name but different formal argument / diff parameter is know as constructor overloading


this Keyword:
Refers to current object
Used when variable name and parameter name are same
Used to call constructor

Super Keyword 
Refers to parent class object
Used to call parent constructor
Used to call parent method

Mathematical Task:
 1.Math.min(x,y)
 2.Math.max(x,y)
 3.Math.sqrt(x)
 
----------------------------------------------------------------------------------------------------------------------------------
Arrays:
------
    1.  Arrays are used to store multiple values in a single variable with same data type 
        why should we need to use arrays?
          -To store multiple values in a single row
          -faster to acces using index
          -easy to loop using for or foreach loop
          -saves memory compare to variables
   ex:int [] arr ={1,2,3,4}


   Types of Arrays:
   ----------------
                  1.One Dimensional Array
                  2.Two-Dimensional Array(2D)
                  3.Multi-Dimensional Array(3D)


   2.We can declare arrays in 2 ways:
         A.int [] arr =[1,2,3,4]
         B.int arr[]={1,2,3}

   3.Creating and Allocating Memory for an Array
        int arr = new int[5];
   
   4.initializing Arrays:
       arr[0]=1;
       arr[1]=2;
       arr[2]=3;

    => other way of initializing
       int [] arr ={1,2,3,4,5}

    => Accessing Array Element:
       System.out.println(arr[1]);//this will give you first Element
      
    => To get Array length
       System.out.prinln(arr.length);



1.How are Arrays store in the Memory?
->An Array is an Object in Java
->It is stored in Heap Memory

2.What happens in memory?
->arr is a reference variable
->Java creates an array object in Heap
->arr stores the address (reference) of that array

3.Important Interview Points 
->Arrays are stored in Heap memory
->Array name is a reference variable
->Index starts from 0 for performance
->Array size is fixed
->Multiple references can point to same array


java.util.Arrays (Utility Class):
---------------------------------
                    -> Java provides the Arrays class to make array operations easy
                    ->import java.util.Arrays;

               sort():
               ------
                    ->It is used to sort in ascending array
                    
               Tostring():
               ---------
                    toString() converts an object into a readable String form.

               equalto():
               ---------
                     -> equalto() it is used to compare values
                     ->a==b checks the object's address
                     ->a equals b method checks the value


            if we have Primitive variable 
            a=b // true
            if we used in array  returns false since it compares address variable not the value

            copyOf():
            ---------
                ->same value different memory is know as copy of method
                int[] boxA = {1, 2, 3};
                int[] boxB = Arrays.copyOf(boxA, boxA.length);

                o/p:
                  boxA → [1, 2, 3]
                  boxB → [1, 2, 3]

             fill():
             ------
                Arrays.fill() is used to fill (assign) the same value to all elements of an array.

            binarySearch():
            -------------
                 Binary Search is a fast searching technique used to find an element in a SORTED array
--------------------------------------------------------------------------------------------------------------------------------------------
static Variable:
            ->Anything that is prefixed with static keyword is know as Static variable
               ex: static String carname = scala;

            ->A static variable belongs to the class, not to objects.

            ->Only ONE copy exists, shared by all objects.

            ->static variable Single memory shared

Non-static Variable:
            ->Anything that is not prefixed with static keyword is know as Non-static Variable
            ex:String carname = scala;
           
            ->Non static variable belongs to the object , not to the class

            ->Non-static New memory for every object








----------------------------------------------------------------------------------------------------------------------------------------------------------------------------------------
----------------------------------------------------------------------------------------------------------------------------------------------------------------------------------------
                                               Java Part- 2
                                               -------------

OOP:
    ->Object Oriented Programming
    ->OOP is a way of writing programs using real-world objects.


Note:A procedural language is a type of programming language where:
why Oops choose over Procedural languvage
Dis-Advantage of Procedural Languvage:
   ->hard to Understand
   ->Hard to Reduces
   ->Hard to Maintain
   ->code becomes big

Oops
   ->easy to Understand
   ->code Reduces
   ->code re-useable
   ->easy to Understand
   ->Less errors


Procedural vs OOP:
------------------
                  
                Procedural Programming
                ->focus on functions
                ->no security for data
                ->Data is seprare

                Oop:
                ->focus on Object
                ->Data +method together
                ->Data is protected




Object :
     ->Anything that has existing in the real world enity
     ->All the object has states (data,properties ,values) and behavior (Action,function,work)
     ->states should be represented by non-static variable
     ->Behaviours should be represented by non-static methods

      1.Can we represent states or behaviours with static ?
      ->Yes,member which is Same for all the Obj should be static
      ->Member which is diff for all the Object should be reprsented by non-stattic

what is a class?
            class is a keyword
            class are blue print of an object
            it defines state and behaviour of an object
            A class can be preffxed with any kind of access Modifier



            Syntax: 
                  public class main{
                     int x =5;
                  }


what is an object?
           An object is an instance of a class that represents a real-world entity and occupies memory.

           ClassName variable = new keyword constructor
           car scala = new car();



               




      

   



          
        
  


   



















