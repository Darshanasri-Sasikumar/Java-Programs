1.view
    A VIEW is a virtual table based on the result of a SQL query.
It does not store data physically, but retrieves data from underlying tables when accessed.
    ->It is a virtual table
    ->it does not take actual memory space
    ->reduces complexity of sql queries
2.index
        ->Index is an data base Object
        ->It improves speed of the retrival data from the table
        ->used for large data
        ->brings an exact matching records
        ->Quick to locate row without exact match to the table

3.cluster vs non-cluster
| Feature          | Clustered Index                                                 | Non-Clustered Index                          |
| ---------------- | --------------------------------------------------------------- | -------------------------------------------- |
| Storage          | **Reorders the actual table rows**                              | Separate structure from the table            |
| Number per table | Only **1 per table** (because rows can be ordered only one way) | Multiple indexes allowed                     |
| Speed            | Faster for **range queries**                                    | Slightly slower than clustered               |
| Use              | Primary key is usually clustered                                | Used on columns that are frequently searched |
| Example          | Table sorted by `emp_id`                                        | Index on `emp_name` to search quickly        |
