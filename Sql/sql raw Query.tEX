1.To create Database :
            Create DataBase DataBaseName;
            Syntax:Create Database Practice;


2.To create Table:
            Create Table TableName(
            ColumnName DataType Constrain,
            ColumnName DataType Constrain,
            ColumnName DataType Constrain);

        Example:
            use practice;
                CREATE TABLE Student (
                    Id INT PRIMARY KEY,
                    FirstName VARCHAR(255),
                    LastName VARCHAR(255),
                    SchoolName VARCHAR(255));

3.To Insert values inside the table
                    INSERT INTO TABLENAME(COLUM1,COLUM2,COLUM3)
                    VALUES('value1','value2','value3');

                    Example:
                    INSERT INTO STUDENT(ID,FIRSTNAME,LASTNAME,SCHOOLNAME)
                    values (1,'DARSHANA','SRI','JV');

4.To update value
                UPDATE TableName,
                Set ColumnName = 'value'.
                where condition;
5.To delete value
                Delete from TableName

6.To  list 
                Select *
                from TableName;

-----------------------------------------------------------------------------------------------------------------------------------------------------
SQL Aggregate Functions
        An aggreate function is to perform calculation,set of values and returns a single value
        There are 5 most commonn sql functions:
                ->min()
                ->max()
                ->count()
                ->sum()
                ->avg()

----------------------------------------------------------------------------------------------------------------------------------------------------------
inner Join->used fetch the matching recorde
left Join->Used to fetch matching records from the left table
Right Join->It is used to fetch matching recorde 
full outter Join->fetches record for all the table



--------------------------------------------------------------------------------------------------------------------------------------------------
DDL->Data Definition Languvage:
            ->It is used to create or change the table structure
            ->Create->create a new table
            ->Alter->Modifies the exisitng table
            ->Drop->deletes the table
            ->Truncate->Keeps the table Deletes the records

DML->Data Manipulation Languvage:
            ->Insert->it is used to insert records inside the table(new data)
            ->Select->It is used to fetch the data
            ->update->modify the existing data
            ->Delete->Removes Specific Data

DQL->Data Querry Languvage:
            ->Select->it is used to select partucular column or retrives all the data

DCL->Data Control Languvage:
            ->Grant->Grants some permission to the users
            ->revoke->

TCl->Transaction Control Languvage:
            ->Commit->saves the changes
            ->rollBack->undo the changes
            ->Save point

TRUNCATE TABLE → removes all rows but keeps the table structure intact.

DELETE → can remove rows with conditions and can be rolled back (if transaction is used).

DROP → removes table and structure completely.


Which SQL keyword is used to combine the result sets of two SELECT queries and remove duplicates?
->UNION combines the result sets of two SELECT queries and removes duplicates.
->UNION ALL → combines results including duplicates.
->JOIN → combines columns from two tables.
->MERGE → used in some SQL dialects for insert/update operations.


Composite Primary Key:
            A composite primary key is a primary key made up of two or more columns to uniquely identify a row.
            It cannot allow NULL values
            It enforces uniqueness across the combination of columns

1.Difference between DBMS and RDBMS
2.what is RDBMS
3.types of DataType:
        ->char->memory space is high -?size 2000
        ->VARCHAR-> variable memory allocation ->size 4000
        ->number->precision (int ) ->scale(decimal)
        ->timestamp
        ->Date
        ->Large Object ->Binary LO 
4.constraints:it is used to validate the column data
                ->uniquely
                ->Not NULL
                ->check
                ->primary key
                ->fk

5.difference between pk and fk

    Primary Key
                ->uniquely
                ->does not accept null values
                ->a table can have only one pk
                ->it is combination of unique and not null

    Foreign Key
                ->A table can have one or more fk
                ->it accepts null
                ->not unique
                ->it is used conect one table to another

6.ddl->Create,Rename,Alter,Drop,Truncate
7.DML->Insert,update,delete
8.TCL ->commit ,Rollback and save point
9.DCL ->Grant and revoke
10.DQL ->select,selection,projection and joins

11.DQL -> 
It is used to fetch the data from the database
->select->retrive the data from table
->selection->retrive both row and column in table
->projection->select onyl data from table by selecting the colums only
->joins->it is used to fetch the data from 2 different table

Note:Expression- Combination of operators and operonds called Expression

12.what is meant by alias
13.where clause:
            ->It is used to filter the records
            ->it executes row by row
            ->It executes after from clause
            ->we can pass multiple conditions

        Order of execution:
            from
            where
            select
            distinct
            orderby
14.what is meant by distinct clause:
        ->Disitict clause is used to remove duplicates or repeated values
        ->It should be taken as first arugument
        
15.Logical Operators:
        AND,OR,Not

16.set Operators
            ->UNION->It is used to combine 2 or more queries  and also  unique
            ->Union All->it is used to retrive all the data 
            ->Intersect
            ->Minus

Group:
        ->Group by clause is used to group the records one or more columns with the help of aggreagte functions
        ->Group by clause we can use column name and Expression
        ->Group by clause excutes row by row
        ->Any conition executes after group by cluse it must execute group by group  
Having clause:
        ->Group filter condition
17.Functions:
            ->it is a set of codes it is been executed when it is been called
            ->we have 2 type of function:
                    +build-in function
                    +user define function 

        Build in function can be further calssified into 
                            -> single row function
                            -> Multi row function

        ->Single row function
                    ->Takes input only 1 at a tine and pocceds the Output
                    ->it executes row by row
                    ->for single row functions will have n number of input and n number of Output

        ->MultiRow Functiona:
                    ->Multi row function is also known as aggreate function or group function
                    ->It executes group by group
                    ->for n number of input provides only one Output
                    type:
                        ->min()
                        ->max()
                        ->sum()
                        ->count()
                        ->avg()

                    Rules of Multi Row functions:
                            ->MRF can accept only one arugument
                            ->along with MRF it should not take any other arugument
                            ->we cannot pass multi row function in where clause
                            ->count is the only function which accepts count(*) as input
                            ->multi row function ignores null value

Single Row Function:
                ->Length()
                ->concat()->||
                ->upper()
                ->lower()
                ->reverse()
                ->initcap()
                ->mod()
                ->round()
                ->truncc()
                ->nvl()
                ->to_char()
                ->month_between()
                ->add_month() 
                ->last_Day()
                ->substr()Returns a portion of a string starting from a given position.
                ->instr()Returns the position of a substring within a string.
                ->replace()
            
difference between instr() and substr()

Having clause:
            ->It works as group filter function
            ->it executes after groupby clause
            ->having clause is depended to the group by clause
            ->we can have multi row function inside the group by clause
