1.To create Database :
            Create DataBase DataBaseName;
            Syntax:Create Database Practice;


2.To create Table:
            Create Table TableName(
            ColumnName DataType Constrain,
            ColumnName DataType Constrain,
            ColumnName DataType Constrain);

        Example:
            use practice;
                CREATE TABLE Student (
                    Id INT PRIMARY KEY,
                    FirstName VARCHAR(255),
                    LastName VARCHAR(255),
                    SchoolName VARCHAR(255));

3.To Insert values inside the table
                    INSERT INTO TABLENAME(COLUM1,COLUM2,COLUM3)
                    VALUES('value1','value2','value3');

                    Example:
                    INSERT INTO STUDENT(ID,FIRSTNAME,LASTNAME,SCHOOLNAME)
                    values (1,'DARSHANA','SRI','JV');

4.To update value
                UPDATE TableName,
                Set ColumnName = 'value'.
                where condition;
5.To delete value
                Delete from TableName

6.To  list 
                Select *
                from TableName;

-----------------------------------------------------------------------------------------------------------------------------------------------------
SQL Aggregate Functions
        An aggreate function is to perform calculation,set of values and returns a single value
        There are 5 most commonn sql functions:
                ->min()
                ->max()
                ->count()
                ->sum()
                ->avg()

----------------------------------------------------------------------------------------------------------------------------------------------------------
inner Join->used fetch the matching recorde
left Join->Used to fetch matching records from the left table
Right Join->It is used to fetch matching recorde 
full outter Join->fetches record for all the table



--------------------------------------------------------------------------------------------------------------------------------------------------
DDL->Data Definition Languvage:
            ->It is used to create or change the table structure
            ->Create->create a new table
            ->Alter->Modifies the exisitng table
            ->Drop->deletes the table
            ->Truncate->Keeps the table Deletes the records

DML->Data Manipulation Languvage:
            ->Insert->it is used to insert records inside the table(new data)
            ->Select->It is used to fetch the data
            ->update->modify the existing data
            ->Delete->Removes Specific Data

DQL->Data Querry Languvage:
            ->Select->it is used to select partucular column or retrives all the data

DCL->Data Control Languvage:
            ->Grant->Grants some permission to the users
            ->revoke->

TCl->Transaction Control Languvage:
            ->Commit->saves the changes
            ->rollBack->undo the changes
            ->Save point

TRUNCATE TABLE → removes all rows but keeps the table structure intact.

DELETE → can remove rows with conditions and can be rolled back (if transaction is used).

DROP → removes table and structure completely.


Which SQL keyword is used to combine the result sets of two SELECT queries and remove duplicates?
->UNION combines the result sets of two SELECT queries and removes duplicates.
->UNION ALL → combines results including duplicates.
->JOIN → combines columns from two tables.
->MERGE → used in some SQL dialects for insert/update operations.


Composite Primary Key:
            A composite primary key is a primary key made up of two or more columns to uniquely identify a row.
            It cannot allow NULL values
            It enforces uniqueness across the combination of columns