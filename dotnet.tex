Middleware:
        Middleware is a component that exists in the request–response pipeline between the client and the server.
          ->Middleware is software that sits between the client request and server response in an ASP.NET Core application.
          ->Middleware Acts as an checkpoint between request and response
            ->Middleware is a component that runs between the request and the response
            ->It is present inside a pipeline Acts As an GateKeeper between Request and responses
            ->pipeline means sequence of middle ware
            ->Used in Authentication and Authorization
            Request  →  Middleware 1 → Middleware 2 → Controller
            Response ←  Middleware 1 ← Middleware 2 ← Controller
            ->Authentication & Authorization
            ->Logging
            ->Exception handling
            ->Request/Response modification
            ->Routing

            Built-in Middleware Examples

UseAuthentication()

UseAuthorization()

UseRouting()

UseExceptionHandler()

UseCors()

UseStaticFiles()


Authentication->checks who you are
Authorization->what are the access you are
Dependency Injection - Dependency Injection is a design pattern where objects are given the things they need, instead of creating them by themselves.



MVC Pattern
-------

MVC stands for Model – View – Controller.
It is a design pattern used to separate application logic, UI, and data.

Model

        Holds database information
        Contains table structure
        Column names
        Data types
        Model is a class
        Supports features like Soft Deletes if required
        Communicates with the database
        It is also used to write the relationship like hasOne and hasMany(how to define na Navigation property->it has virtual keyword)
        Validation can be given as Annotation

Validation:
            given as Annotation
            or Fluent Api DB Context

Eager Loading:
            ->Loading all the related Datas along with main data in a single query or command using "include()"and .ThenInclude()

Lazy Loading:
            ->Loading Related Data only when it is called is lazy Loading

Expecilcit Loading:
            ->it is similar to LazyLoading but we should initiate related Data manually

📌 Example: Entity classes, data validation, database mapping

View

            Represents the User Interface (UI)
            Displays data to the user
            Uses technologies like:
            HTML
            CSS
            JavaScript
            jQuery
            Vue.js

Does not contain business logic

📌 Example: Pages, forms, tables, UI screens

Controller

                Acts as a bridge between Model and View
                Handles HTTP requests and responses
                Receives user input
                Calls Model to process data
                Sends data to View for display

📌 Example: Handles GET, POST, PUT, DELETE requests

MVC Flow

            User sends a request
            Controller receives the request
            Controller interacts with Model
            Model returns data
            Controller passes data to View
            View displays the result to the user

Rest Api:
REST API is a stateless communication mechanism that allows client and server to exchange data using HTTP methods and JSON format



Got it 👍
Since you have **3 days** and this is a **.NET Analyst interview**, I’ll give you a **clear, practical plan** (not overloaded) and focus on **commonly asked topics**.

I’ll also explain *why* each topic matters, in **simple English**, as you prefer.

---

## 🎯 What a .NET Analyst Interview Usually Tests

They check:

* Your **.NET & C# fundamentals**
* Your **understanding of application flow**
* Your **database + SQL skills**
* Your **basic system & analysis thinking**
* Your ability to **explain clearly** (very important for Analyst role)

---

# 📅 3-Day Preparation Plan

---

## 🟢 DAY 1 – C# & .NET CORE FUNDAMENTALS (Very Important)

### 1️⃣ C# Basics (Must Prepare)

✔ OOP Concepts

* Class & Object
* **Encapsulation**
* **Inheritance**
* **Polymorphism**
* **Abstraction**

> *Example questions:*

* Difference between abstract class and interface
* Method overloading vs overriding

✔ Keywords

* `static`
* `virtual`
* `override`
* `sealed`
* `readonly`
* `const`

✔ Value types vs Reference types

* `int`, `struct` vs `class`, `object`

---

### 2️⃣ .NET Framework / .NET Core

✔ Difference between:

* .NET Framework vs .NET Core vs .NET 6/7/8
* MVC vs Web API

✔ Application Flow

```
Request → Controller → Service → Repository → Database
```

✔ Dependency Injection (VERY COMMON)

* Constructor Injection
* Why DI is used

---

### 3️⃣ Exception Handling

✔ `try`, `catch`, `finally`
✔ Custom Exceptions
✔ `throw` vs `throw ex`

---

## 🟢 DAY 2 – ASP.NET + SQL (High Weightage)

---

## 🔵 ASP.NET Core / MVC / Web API

✔ MVC Architecture

* Model
* View
* Controller

✔ REST API Concepts

* GET
* POST
* PUT
* DELETE
* Status codes (200, 201, 400, 401, 404, 500)

✔ Routing

* Attribute routing

✔ Model Validation

* `[Required]`
* `[StringLength]`
* `[Range]`

✔ Authentication & Authorization (Basic)

* JWT (what & why)
* Role-based access

---

## 🟣 SQL (VERY IMPORTANT for Analyst)

Prepare **hands-on concepts**:

✔ Basic Queries

```sql
SELECT, INSERT, UPDATE, DELETE
```

✔ WHERE, LIKE, BETWEEN, IN

✔ JOINS

* INNER JOIN
* LEFT JOIN
* RIGHT JOIN

✔ GROUP BY & HAVING

✔ Aggregate Functions

* COUNT
* SUM
* AVG
* MAX
* MIN

✔ Primary Key & Foreign Key

✔ Normalization (1NF, 2NF – basic idea)

> *Common question:*
> **Difference between WHERE and HAVING**

---

## 🟢 DAY 3 – ANALYST SKILLS + REAL QUESTIONS

---

## 🔶 Analyst / Business Understanding (Very Important)

✔ What does a **.NET Analyst** do?

* Understand client requirements
* Analyze data & workflows
* Convert business needs into technical solutions
* Coordinate between **client & developer**

✔ Requirement Gathering

* Functional vs Non-Functional requirements

✔ SDLC

* Requirement
* Design
* Development
* Testing
* Deployment
* Maintenance

✔ Basic Agile / Scrum

* Sprint
* Daily stand-up
* Product backlog

---

## 🔴 Frequently Asked Interview Questions

Prepare answers for:

1️⃣ Explain OOP concepts with real example
2️⃣ Difference between MVC and Web API
3️⃣ How does a request reach the database in ASP.NET
4️⃣ What is dependency injection?
5️⃣ Explain JOIN with example
6️⃣ What is JWT authentication?
7️⃣ How do you handle exceptions in .NET?
8️⃣ How do you analyze a requirement from client?
9️⃣ What challenges have you faced in projects?
10️⃣ Why should we hire you as a .NET Analyst?

---

## 📝 Bonus (If Time Left)

✔ LINQ (basic)

* `Where`
* `Select`
* `FirstOrDefault`

✔ Entity Framework Core

* DbContext
* DbSet
* Migrations

---

## 🧠 Interview Tip (Very Important)

You **don’t need to be perfect**.
But you must:

* Explain **clearly**
* Use **simple examples**
* Show **logical thinking**

If you want, I can:

* ✅ Give **daily revision notes**
* ✅ Share **mock interview questions**
* ✅ Give **short answers you can memorize**
* ✅ Create a **last-day revision checklist**


difference between for and foreachloop
out and drift keyword
cluster and non cluster
enna pattern 
inhertance 
.NET
ASP
.net and core framework difference
web API
Relationship
Join
what is refference what is value


Middleware
Constructor
Difference between Abstract

Dependency Injection
life spans
loosely coupled 


WebApi:
        Web API is used to build HTTP-based services.
        It supports HTTP methods like GET, POST, PUT, PATCH, DELETE.
        Data is passed through URL parameters, query strings, headers, or request body.
        It returns data, usually in JSON format.
        It is used by web applications, mobile applications, and other systems

Dependency Injection:
                    Dependency Injection is nothing but utilising the object from somewhere but not actually using it
                    the object it dependent from other class
                    it does not create a new class 
                    Since new object is not create there it takes less space in memory location

we have 3 types of DI:
                    Transient (AddTransient)->New instance of the service is created when ever requested
                    Scoped -> A single tone class is created for every scope
                    Singleton (AddSingleton)->A single instance is created for entier application


asynch and await
                    Async->it does not wait for other methods to execute it parallely gets executed
                    it is used in the start in the start of the program
                    Multiple async methods can run in parallel.



await:
                    Await function waits till task to be completed without disturbing the main thread
                    it is used inside the asynchronus program
                    it executes one by one without killing the main thread

Entity Framework:
-----------------
                    Enitiy frame work is a Object Relational Mapping
                    where No raw querry to be written 
                    support multiple database
                    Handles DataBase Connection
                    Supports LINQ to query data
                    saves time and reduce human errors
                    supporst relationship:
                    one to one 
                    one to many
                    many to one
                    many to many
                    

Keywords:
        1.static :
         ->Anything that is belongs to the class not the Object 
         ->It can be Prefixed with method or varable
         ->it can be declared inside the class not to the method
         ->It can be Accessed without creating and Object
         ->static method cannot be overrided
         class parent{
                public static main(){
                        Console.WriteLine('hi');
                }
         }

         2.Virtual :
                ->It can be prefixed me with method ,proprties
                ->it can be override from parent class to child class
                ->Basically it is used for runtime
                ->It must be declared inside the class
                ->it can be override a method with override keyword
                class parent(){
                        public virtual method(){
                                console.WriteLine("hi");
                        }
                }

                class child : parent(){
                        public override method(){
                                Console.WriteLine("bye");
                        }
                }
                class Program
                {
                static void Main()
                {
                        Animal a = new Dog(); // base class reference, derived class object
                        a.MakeSound();        // Output: Dog barks
                }
                }

        3.override:
                        ->it is used in child class
                        ->it allows child method to override
                        ->the parent method must be virtual

        4.Sealed:
                        ->static method cannot be sealed
                        ->sealed method can be Inheried
                        ->override sealed method cannot be inherited
                        ->sealed cannot use virtual directly
                        ->No child no further challenges
        5.readonly:
                        ->does not allow to change
                        ->it can be set onlt once 
                        ->the value of readonly be constant
                        ->it can be used only at the time declaration or inside the ccontroller
                        ->it can be write only once
                        ->const is fixed forever

Value type:
                value type is nothing but it stores actual values
                when u copy it creates a new copy
                Changes in one do not affect the other
                ->int
                ->char
                ->double
                ->float
                ->boolean
                ->struct
                ->enum

refference type:
                it is Usually store the address of the value
                when u copy it both contain same address
                Changes in one affect the other
                ->class
                ->interface
                ->string
                ->array
                ->object
                    

| Point               | Abstract Class                            | Interface                              |
| ------------------- | ----------------------------------------- | -------------------------------------- |
| Variables           | Can have variables                        | Cannot have variables (only constants) |
| Access Modifiers    | Can use `public`, `protected`, `private`  | All methods are `public` by default    |
| Fields              | Can have instance fields                  | Only `public static final` (constants) |
| Method Type         | Can have abstract + concrete methods      | Only abstract methods (basic level)    |
| Inheritance Keyword | `:` (inherits)                            | `:` (implements)                       |
| Purpose             | Used when classes are **closely related** | Used when classes are **not related**  |
| Speed               | Slightly faster                           | Slightly slower                        |
| State (data)        | Can store state (data)                    | Cannot store state                     |
|Constructor          |it can have Constructor                    | It does not have any Constructor
|                      implementation can be done with abstract     interface can have implementation to the child class
                        and concrete method   
                        multiple inhertance is not possible             multiple inhertance is not possible     
                        
                        
DbContext:
        ->Db context is a class provided by entityframeworkcore
        ->it acts as an bridge between C# and database
        ->It Represents the data base
        ->allows to querry and save the database
        ->it also handles db transactions 
        ->It contain DB set

DBset Represents set in database
DbSet<T> = Represents a table
DbContext = Represents the database
SaveChanges() = Saves data to the database

Data Mapping :
        ->converting table data into Object
        ->As it Reduces Manual COading
        ->To maintain Application clean and maintable
        ->It supports ORM
        Types of Data Mapping
1️⃣ Manual Data Mapping
        Developer assigns values manually
        Example: user.Name = row["name"];
2️⃣ Automatic Data Mapping
        Framework maps automatically
        Example: Entity Framework

        Advantages of Data Mapping
        Less code
        Less errors
        Easy maintenance
        Clean architecture


Destructor:

        A destructor is a special method
        It is used to fress out the memory to fress out the resoure
        It is used to release unmanaged resources
        It is called automatically by the Garbage Collector
        It does not delete objects itself
        Garbage Collector (GC):
        GC is responsible for memory management
        It identifies unused objects
        It removes objects from memory
        It may call the destructor before destroying the object


Difference between const and readonly

| Feature                | `const`                              | `readonly`                               |
| ---------------------- | ------------------------------------ | ---------------------------------------- |
| Value change           | ❌ Cannot be changed                  | ✅ Can be changed **only in constructor** |
| When value is assigned | **Compile time**                     | **Run time**                             |
| Inheritance            | Value is **copied** to derived class | Value is **inherited**                   |
| Access                 | Always **static implicitly**         | Can be **instance or static**            |
| Modification           | Never allowed                        | Allowed only in constructor              |

Difference between thow and throws

| Feature              | **throw**                                 | **throws**                          |
| -------------------- | ----------------------------------------- | ----------------------------------- |
| Purpose              | Used to **explicitly throw an exception** | Used to **declare exceptions**      |
| Used in              | Method body                               | Method declaration                  |
| Language             | Java & C#                                 | **Java only**                       |
| Number of exceptions | Throws **one exception at a time**        | Can declare **multiple exceptions** |
| Handling             | Transfers control to `catch`              | Passes responsibility to caller     |



Ref :
        -ref is used to pass a variable by reference.
        A reference variable points to the same memory location instead of creating a copy.
        It allows the method to read and modify the original variable.
        It is used to modify existing memory values.
        The variable must be initialized before passing.
        Any changes made inside the method are reflected in the original variable.

out Keyword

                Used to pass a variable by reference for output purpose.
                Points to the same memory location.
                Method cannot read the value before assigning.
                Used to return multiple values from a method.
                Variable need not be initialized before passing.
                Method must assign a value before returning.
                Changes inside the method reflect in the original variable.

ref requires initialization and is used to modify an existing value, whereas out does not require initialization and is used to return a value from a method.

Cluster:
        A cluster is a connect multiple database server working together as one databse serve
        It defines the physical order of the database
        Only one cluster index per Pages
        Stored in a table order
        only one cluster is allowed in a table
        faster for range queries
        data pages store in sorted order



Non-cluster:
        ->It does not have any physical order insted it has logical order
        ->store in seperate index structure using pointers 
        ->Mulitiple non cluster allows in a table
        faster for lookup queries
        Stored in a separate index structure.


Store Procedure:
        ->A storeProcedure is a collection of SQL statements
        ->It is used to perform sequence of operations such as querring,insert,update and delete
        ->Unlike regular SQL queries, stored procedures group multiple SQL statements into a single reusable unit, avoiding repeated SQL code.
  The benfits of store procedure:
                ->Code reusability
                ->Enhance performance
                ->security
        Syntax:
                CREATE PROCEDURE ProcedureName
                @Column1 DataType
                AS
                BEGIN
                -- SQL statements
                END;

Trigger:
        Trigger is a database object 
        Trigger is used to perform SQL operations
        it is been fired when  DML statemt like ->insert update delete can be perfomed
       Triggers are automatically invoked; they do not need to be called explicitly.
        
  dis-Adv:
        -> Hard to understand
        ->some of the code are hidden to the developer
        ->can affect performance


LINQ:
        Languvage Integrated queries
        It is used to perform querry using c# Languvage
        it acts as an bridge between c# and database
        it is used to querry,filter,sort 
        Works with collections, databases, XML, and objects.
        Type-safe and easy to read.
        Reduces complex loops and conditions.


Delegates :
       Delegates defines the method signature that it points To
       it can reffer both static and non static method
       Delegates are type-safe meaning the method signature must match the delegate declaration.

  when to use?
        To achieve loose coupling
        For event handling
        It is use to handle the exception
        It is used to write flexible and reueable code
        for Functional style program we can use LINQ function

Declaration of Delegates 
        access_modifier delegate return_type DelegateName(parameter_list);

using System;
public class DelegateExample
{
    // Delegate declaration
    public delegate void MyDelegate(string message);

    // Method matching the delegate signature
    public static void DisplayMessage(string msg)
    {
        Console.WriteLine("Message: " + msg);
    }

    public static void Main()
    {
        // Instantiating delegate
        MyDelegate del = DisplayMessage;

        // Invoking delegate
        del("Hello from delegate!");
    }
}

what is meant by event handlers:


Difference between = and == and equals()
        = assigns a value ,operator,it cannot be ovveriden
        ==checks the value or memory address
        equals()checks the value,method ,can be overriden

Throw vs Throwex
| Point         | `throw`                          | `throw ex`                    |
| ------------- | -------------------------------- | ----------------------------- |
| Purpose       | Re-throws the **same exception** | Throws the exception again    |
| Stack trace   | **Keeps original error line**    | **Loses original error line** |
| Debugging     | Easy to find real issue          | Hard to find real issue       |
| Best practice | ✅ **Recommended**                | ❌ Not recommended             |
| Used inside   | `catch` block                    | `catch` block                 |
