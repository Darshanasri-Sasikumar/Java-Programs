Middleware:
          ->Middleware Acts as an checkpoint between request and response
            ->Middleware is a component that runs between the request and the response
            ->It is present inside a pipeline Acts As an GateKeeper between Request and responses
            ->pipeline means sequence of middle ware
            ->Used in Authentication and Authorization
            Request  →  Middleware 1 → Middleware 2 → Controller
            Response ←  Middleware 1 ← Middleware 2 ← Controller
            ->Authentication & Authorization
            ->Logging
            ->Exception handling
            ->Request/Response modification
            ->Routing

            Built-in Middleware Examples

UseAuthentication()

UseAuthorization()

UseRouting()

UseExceptionHandler()

UseCors()

UseStaticFiles()


Authentication->checks who you are
Authorization->what are the access you are
Dependency Injection - Dependency Injection is a design pattern where objects are given the things they need, instead of creating them by themselves.



MVC Pattern
-------

MVC stands for Model – View – Controller.
It is a design pattern used to separate application logic, UI, and data.

Model

        Holds database information
        Contains table structure
        Column names
        Data types
        Model is a class
        Supports features like Soft Deletes if required
        Communicates with the database
        It is also used to write the relationship like hasOne and hasMany(how to define na Navigation property->it has virtual keyword)
        Validation can be given as Annotation

Validation:
            given as Annotation
            or Fluent Api DB Context

Eager Loading:
            ->Loading all the related Datas along with main data in a single query or command using "include()"and .ThenInclude()

Lazy Loading:
            ->Loading Related Data only when it is called is lazy Loading

Expecilcit Loading:
            ->it is similar to LazyLoading but we should initiate related Data manually

📌 Example: Entity classes, data validation, database mapping

View

            Represents the User Interface (UI)
            Displays data to the user
            Uses technologies like:
            HTML
            CSS
            JavaScript
            jQuery
            Vue.js

Does not contain business logic

📌 Example: Pages, forms, tables, UI screens

Controller

                Acts as a bridge between Model and View
                Handles HTTP requests and responses
                Receives user input
                Calls Model to process data
                Sends data to View for display

📌 Example: Handles GET, POST, PUT, DELETE requests

MVC Flow

            User sends a request
            Controller receives the request
            Controller interacts with Model
            Model returns data
            Controller passes data to View
            View displays the result to the user

Rest Api:
REST API is a stateless communication mechanism that allows client and server to exchange data using HTTP methods and JSON format



Got it 👍
Since you have **3 days** and this is a **.NET Analyst interview**, I’ll give you a **clear, practical plan** (not overloaded) and focus on **commonly asked topics**.

I’ll also explain *why* each topic matters, in **simple English**, as you prefer.

---

## 🎯 What a .NET Analyst Interview Usually Tests

They check:

* Your **.NET & C# fundamentals**
* Your **understanding of application flow**
* Your **database + SQL skills**
* Your **basic system & analysis thinking**
* Your ability to **explain clearly** (very important for Analyst role)

---

# 📅 3-Day Preparation Plan

---

## 🟢 DAY 1 – C# & .NET CORE FUNDAMENTALS (Very Important)

### 1️⃣ C# Basics (Must Prepare)

✔ OOP Concepts

* Class & Object
* **Encapsulation**
* **Inheritance**
* **Polymorphism**
* **Abstraction**

> *Example questions:*

* Difference between abstract class and interface
* Method overloading vs overriding

✔ Keywords

* `static`
* `virtual`
* `override`
* `sealed`
* `readonly`
* `const`

✔ Value types vs Reference types

* `int`, `struct` vs `class`, `object`

---

### 2️⃣ .NET Framework / .NET Core

✔ Difference between:

* .NET Framework vs .NET Core vs .NET 6/7/8
* MVC vs Web API

✔ Application Flow

```
Request → Controller → Service → Repository → Database
```

✔ Dependency Injection (VERY COMMON)

* Constructor Injection
* Why DI is used

---

### 3️⃣ Exception Handling

✔ `try`, `catch`, `finally`
✔ Custom Exceptions
✔ `throw` vs `throw ex`

---

## 🟢 DAY 2 – ASP.NET + SQL (High Weightage)

---

## 🔵 ASP.NET Core / MVC / Web API

✔ MVC Architecture

* Model
* View
* Controller

✔ REST API Concepts

* GET
* POST
* PUT
* DELETE
* Status codes (200, 201, 400, 401, 404, 500)

✔ Routing

* Attribute routing

✔ Model Validation

* `[Required]`
* `[StringLength]`
* `[Range]`

✔ Authentication & Authorization (Basic)

* JWT (what & why)
* Role-based access

---

## 🟣 SQL (VERY IMPORTANT for Analyst)

Prepare **hands-on concepts**:

✔ Basic Queries

```sql
SELECT, INSERT, UPDATE, DELETE
```

✔ WHERE, LIKE, BETWEEN, IN

✔ JOINS

* INNER JOIN
* LEFT JOIN
* RIGHT JOIN

✔ GROUP BY & HAVING

✔ Aggregate Functions

* COUNT
* SUM
* AVG
* MAX
* MIN

✔ Primary Key & Foreign Key

✔ Normalization (1NF, 2NF – basic idea)

> *Common question:*
> **Difference between WHERE and HAVING**

---

## 🟢 DAY 3 – ANALYST SKILLS + REAL QUESTIONS

---

## 🔶 Analyst / Business Understanding (Very Important)

✔ What does a **.NET Analyst** do?

* Understand client requirements
* Analyze data & workflows
* Convert business needs into technical solutions
* Coordinate between **client & developer**

✔ Requirement Gathering

* Functional vs Non-Functional requirements

✔ SDLC

* Requirement
* Design
* Development
* Testing
* Deployment
* Maintenance

✔ Basic Agile / Scrum

* Sprint
* Daily stand-up
* Product backlog

---

## 🔴 Frequently Asked Interview Questions

Prepare answers for:

1️⃣ Explain OOP concepts with real example
2️⃣ Difference between MVC and Web API
3️⃣ How does a request reach the database in ASP.NET
4️⃣ What is dependency injection?
5️⃣ Explain JOIN with example
6️⃣ What is JWT authentication?
7️⃣ How do you handle exceptions in .NET?
8️⃣ How do you analyze a requirement from client?
9️⃣ What challenges have you faced in projects?
10️⃣ Why should we hire you as a .NET Analyst?

---

## 📝 Bonus (If Time Left)

✔ LINQ (basic)

* `Where`
* `Select`
* `FirstOrDefault`

✔ Entity Framework Core

* DbContext
* DbSet
* Migrations

---

## 🧠 Interview Tip (Very Important)

You **don’t need to be perfect**.
But you must:

* Explain **clearly**
* Use **simple examples**
* Show **logical thinking**

If you want, I can:

* ✅ Give **daily revision notes**
* ✅ Share **mock interview questions**
* ✅ Give **short answers you can memorize**
* ✅ Create a **last-day revision checklist**

Just tell me 👍

difference between for and foreachloop
out and drift keyword
cluster and non cluster
enna pattern 
inhertance 
.NET
ASP
.net and core framework difference
web API
Relationship
Join
what is refference what is value


Middleware
Constructor
Difference between Abstract

Dependency Injection
life spans
loosely coupled 


Dependency Injection:
                    Dependency Injection is nothing but utilising the object from somewhere but not actually using it
                    the object it dependent from other class
                    it does not create a new class 
                    Since new object is not create there it takes less space in memory location

we have 3 types of DI:
                    Transient (AddTransient)->New instance of the service is created when ever requested
                    Scoped -> A single tone class is created for every scope
                    Singleton (AddSingleton)->A single instance is created for entier application


asynch and await
                    Async->it does not wait for other methods to execute it parallely gets executed
                    it is used in the start in the start of the program
                    Multiple async methods can run in parallel.



await:
                    Await function waits till task to be completed without disturbing the main thread
                    it is used inside the asynchronus program
                    it executes one by one without killing the main thread

Entity Framework:
-----------------
                    Enitiy frame work is a Object Relational Mapping
                    where No raw querry to be written 
                    support multiple database
                    Handles DataBase Connection
                    Supports LINQ to query data
                    saves time and reduce human errors
                    supporst relationship:
                    one to one 
                    one to many
                    many to one
                    many to many
                    
                    

